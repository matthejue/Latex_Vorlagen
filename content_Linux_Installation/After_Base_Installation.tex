%!Tex Root = ../Linux_Installation.tex
% ./content_Linux_Installation/Packete
% ./content_Linux_Installation/Design
% ./content_Linux_Installation/Deklarationen
% ./content_Linux_Installation/Before_Installation
% ./content_Linux_Installation/Base_Installation

\section{Arch Linux After Base Installation}

\begin{frame}[fragile]{General}
  \begin{itemize}
    \item if one forgot one step in the base installation with \inlinebox{su}, one can get root again.
    \item {\tiny\inlinebox{sudo pacman -S base-devel, xorg-xkill, man-db texinfo openssh e2fsprogs, dialog}}: \inlinebox{base-devel} is for building aur packages and \inlinebox{sudo} and \inlinebox{which} are in there, enable \inlinebox{openssh} with \inlinebox{systemsctl enable sshd}, \inlinebox{dialog} is a cli-textbox some programs use.
    \item if sth. goes wrong with the DE one can change tty with \inlinebox{ctrl + alt + fX} and make e.g. \inlinebox{killall i3}.
  \end{itemize}
\end{frame}

\begin{frame}[fragile, allowframebreaks]{Desktop-Environment / WM\vspace{0.5cm}}
  \begin{itemize}
    \item \inlinebox{sudo pacman -S xorg-server xorg-xinit}
    \item \alert{i3:}
      \begin{itemize}
        \item \inlinebox{sudo pacman -S i3-gaps i3status alacritty dmenu}
        \item install fonts (i3 doesn't pull fonts), e.g. \inlinebox{sudo pacman -S noto-fonts}
      \end{itemize}
    \item \alert{xfce:}
      \begin{itemize}
        \item \inlinebox{sudo pacman -S xfce4, xfce4-goodies, lightdm, lightdm-gtk-greeter}, \inlinebox{systemctl enable lightdm}
      \end{itemize}
    \item \alert{gnome:} \inlinebox{pacman -S gnome}, \inlinebox{gnome-tweaks}, \inlinebox{systemctl enable gdm}
    \item \alert{kde:}  \inlinebox{plasma-meta}, \inlinebox{kde-applications}, \inlinebox{systemctl enable sddm}
    \item \inlinebox{cp /etc/X11/xinit/xinitrc /home/areo/.xinitrc}
    \item \inlinebox{nvim ~/.xinitrc}: write \inlinebox{exex i3} or \inlinebox{exec xfce4-session} in there
    \item \inlinebox{startx} to start
    \item \inlinebox{xrandr} to show all available screen resolutions and then e.g. \inlinebox{xrandr -s 1920x780}
  \end{itemize}
\end{frame}

\begin{frame}[fragile]{Start DE directly after login or set up a display manager (login screen)}
  \begin{itemize}
    \item \inlinebox{~/.zshrc} or \inlinebox{~/.bash_profile}:
      \begin{terminal}[minted language=bash]
        if [[ "$(tty)" = "/dev/tty1" ]]; then
          pgrep || startx
        fi
      \end{terminal}
    \item \alert{displaymanager:}
      \begin{itemize}
        \item \inlinebox{sudo pacman -S lightdm lightdm-gtk-greeter}
        \item \inlinebox{sudo systemctl enable lightdm.service}: systemd command to tell systemd to start lightdm when one does log in
        \item useful to be able to choose between differnt desktop environments
      \end{itemize}
  \end{itemize}
\end{frame}

\begin{frame}[fragile]{Compiling yay (make arch package)}
  \begin{itemize}
    \item \inlinebox{git clone https://aur.archlinux.org/yay-git.git}
    \item \inlinebox{cd yay-git} and then \inlinebox{makepkg -si}
      \begin{itemize}
        \item \inlinebox{base-devel} needed for it
      \end{itemize}
  \end{itemize}
\end{frame}

\begin{frame}[fragile]{Arch in Virtualbox (in case)}
  \begin{itemize}
    \item \inlinebox{pacman -S virtualbox-guest-utils xf86-video-vmware}
  \end{itemize}
\end{frame}

\begin{frame}[fragile]{Wifi}
  \begin{itemize}
    \item NetworkManager manages everything ones it is activated (ethernet an wifi)
      \item \inlinebox{wifi-menu} doesn't work once the NetworkManager is activated or if there's already a ethernet connection
    \item \inlinebox{nmcli device wifi list}
    \item {\tiny \inlinebox{nmcli device wifi connect 'FRITZ!Box Gastzugang Herbert' password PASSWORD}}
  \end{itemize}
\end{frame}

\begin{frame}[fragile]{CPU/GPU}
  \begin{itemize}
    \item \alert{Microcode:} \inlinebox{pacman -S amd-ucode} or \inlinebox{pacman -S intel-ucode}
    \item \inlinebox{pacman -S xf86-video-intel}
    \item \inlinebox{pacman -S mesa} (if intel or amd for graphics) or \inlinebox{pacman -S nvidea nvidea-utils} (nvideo for graphics) and \inlinebox{pacman -S nvidea-lts} (if one installed the lts-kernel)
    \item \inlinebox{pacman -S virtualbox-guest-utils xf86-video-vmware} and \inlinebox{systemctl enable vboxservice}(if in Virtualbox)
  \end{itemize}
\end{frame}

\begin{frame}[fragile,allowframebreaks]{Right Keyboard Layout in Xorg\vspace{0.5cm}}
  \begin{itemize}
    \item for xorg the keyboard layout isn't related to the keyboard layout in the tty with it's file: \inlinebox{/etc/vconsole.conf} but has to be configured in e.g. \inlinebox{/etc/X11/xorg.conf.d/00-keyboard.conf} (one of many keyboard layouts for xorg)
      \begin{itemize}
        \item xorg.conf is parsed by the X server at start-up. To apply changes, restart X
      \end{itemize}
    \item \alert{get overview:}
      \begin{terminal}[minted language=bash]
        localectl list\itemx11-keymap-models
        localectl list\itemx11-keymap-layouts
        localectl list\itemx11-keymap-variants [layout] (e.g. de)
        localectl list\itemx11-keymap-options
      \end{terminal}
  \item set one for the current session: \inlinebox{sudo setxkbmap de nodeadkeys} or \inlinebox{sudo setxkbmap -layout de -variant nodeadkeys} (long variant)
    \begin{itemize}
      \item {\tiny \inlinebox{setxkbmap [-model xkb_model] [-layout xkb_layout] [-variant xkb_variant] [-option xkb_options]}}
      \item or persistent in \inlinebox{~/.xinitrc}
    \end{itemize}
  \item make persistent in \inlinebox{/etc/X11/xorg.conf.d}:
    \begin{itemize}
      \item \inlinebox{localectl set-x11-keymap de "" nodeadkeys ""}: autogenerates the keyboard layout file
      \item {\tiny \inlinebox{localectl [--no-convert] set-x11-keymap layout [model [variant [options]]]}}
      \item if \inlinebox{--no-convert} option is passed, the specified keymap is also converted to the closest matching console keymap and applied to the console configuration in \inlinebox{vconsole.conf}
      \item to set a model, variant or options, all preceding fields need to be specified, but the preceding fields can be skipped by passing an empty string with ""
    \end{itemize}
  \end{itemize}
\end{frame}

\begin{frame}[fragile]{Desktop Background}
  \begin{itemize}
    \item {\tiny \inlinebox{feh --bg-scale "/home/areo/Pictures/Wallpaper/linux wallpaper/urban-1597922375998-8560.jpg"}}
    \begin{itemize}
      \item best into \inlinebox{~/.xinitrc}
    \end{itemize}
  \end{itemize}
\end{frame}

\begin{frame}[fragile]{Sound}
  \begin{itemize}
    \item \inlinebox{sudo pacman -S pulseaudio}
    \item \inlinebox{/usr/bin/start-pulseaudio-x11}
    \begin{itemize}
      \item best into \inlinebox{~/.xinitrc}
    \end{itemize}
    \item \inlinebox{pavucontrol} is a gui to have an overview
  \end{itemize}
\end{frame}

\begin{frame}[fragile]{Compositor}
  \begin{itemize}
    \item \inlinebox{picom &}
      \begin{itemize}
        \item best into \inlinebox{~/.xinitrc}
      \end{itemize}
  \end{itemize}
\end{frame}

\begin{frame}[fragile]{Screen-Brightness}
  \begin{itemize}
    \item \inlinebox{sys/class/backlight}
  \end{itemize}
\end{frame}

\begin{frame}[fragile]{Screenshot}
  \begin{itemize}
    \item \inlinebox{scrot} ($\rightarrow$ configuration in \inlinebox{~/.config/i3/config} file)
  \end{itemize}
\end{frame}

\begin{frame}[fragile]{SysRq-Key einsetzen}
  \begin{itemize}
    \item \alert{reboot:} \key{alt} + \key{print} + each of \smalltt{reisub}.
    \item \alert{shut down:} \key{alt} + \key{print} + each of \smalltt{reisuo}.
    \item Bedeutung der Keys kann hier nachgelesen werden: \url{https://en.wikipedia.org/wiki/Magic_SysRq_key}.
    % \item https://en.wikipedia.org/wiki/Magic_SysRq_key
  \end{itemize}
  \begin{Sidenote}
    \begin{itemize}
      \item Nach dem Auslösen von e sollte man den Prozessen ein paar Sekunden Zeit lassen, der Aufforderung, sich sauber zu beenden, nachzukommen.
      \item SysRq may be released before pressing the command key, as long as Alt remains held down.
      \item this keys are for the querty keyboard.
    \end{itemize}
  \end{Sidenote}
\end{frame}

\begin{frame}[fragile]{SysRq-Key aktivieren}
  \begin{itemize}
    \item \alert{direkt aktivieren, aber nicht persistent} {\tiny \inlinebox{echo "1" | sudo tee /proc/sys/kernel/sysrq}}.
    \item \alert{persistent aktivieren} {\tiny \inlinebox*{echo kernel.sysrq=1 | sudo tee /etc/sysctl.d/99-sysctl.conf}.}
  \end{itemize}
  % \item https://linuxconfig.org/how-to-enable-all-sysrq-functions-on-linux
  % \item https://archived.forum.manjaro.org/t/how-to-permanently-enable-the-sysrq-key/137650/6
  \begin{Sidenote}
    \begin{itemize}
      \scriptsize
      \item \alert{geht auch:} {\tiny {\tiny \inlinebox*{sudo bash -c "echo kernel.sysrq=1 > /etc/sysctl.d/99-sysctl.conf"}}}.
      \item {\tiny \inlineboxsout*{sudo echo "kernel.sysrq=1" > /etc/sysctl.d/99-sysctl.conf}}.
      \begin{itemize}
        \item The reason it doesn't work is that ones gives root privileges to echo, which it doesn't need to print to stdout. It's bash doing the writing to file and that's running under your user.
      \end{itemize}
      \item \inlinebox{tee -a} or \inlinebox{>>} for appending.
    \end{itemize}
  \end{Sidenote}
\end{frame}

\begin{frame}[fragile, t]{SysRq-Key checken}
  % , squeeze
  \begin{columns}
    \begin{column}[t]{0.5\textwidth}
      \begin{itemize}
        \item \inlinebox{cat /proc/sys/kernel/sysrq}
          \begin{itemize}
            \item 0 - disable sysrq completely.
            \item 1 - enable all functions of sysrq.
            \item 2 - enable control of console logging level.
            \item 4 - enable control of keyboard (SAK, unraw).
            \item 8 - enable debugging dumps of processes etc.
            \item 16 - enable sync command.
            \item 32 - enable remount read-only.
            \item 64 - enable signaling of processes (term, kill, oom-kill).
          \end{itemize}
      \end{itemize}
    \end{column}
    \begin{column}[t]{0.5\textwidth}
      \begin{itemize}
        \item 128 - allow reboot/poweroff.
        \item 256 - allow nicing of all RT tasks.
      \end{itemize}
      \vspace{0.3cm}
      \begin{Sidenote}
        \begin{itemize}
         \scriptsize
          \item 438 is obtained from the sum of 2 + 4 + 16 + 32 + 128 + 256, so all the corresponding functions are enabled.
        \end{itemize}
      \end{Sidenote}
    \end{column}
  \end{columns}
\end{frame}

\begin{frame}[fragile]{Restore boobtable USB-Stick to normal}{Explanation}
  \begin{itemize}
    \item if one writes a iso-image onto a flash drive there're e.g 2 partitions encoded in the iso image and a lot of free space.
    \item if one writes a iso to the flash drive, it will get a \alert{boot flag} (that can be seen with \inlinebox{sudo fdisk -l}. If one only formats it, it won't work correctly (can't remove partitions etc.).
    \item one has to wipe filesystem completely from the flash drive, to restore it to it's original state.
  \end{itemize}
\end{frame}

\begin{frame}[fragile, allowframebreaks]{Restore boobtable USB-Stick to normal}{Format / repartition a storage device\vspace{0.5cm}}
  \begin{itemize}
    \item \inlinebox{sudo fdisk -l}
    \item \inlinebox{lsblk -o NAME,FSTYPE,SIZE,MOUNTPOINT,LABEL}
    \item \inlinebox{sudo wipefs --all /dev/sdc}
    \begin{itemize}
      \item whole drive, not \inlinebox{sdc1}!
    \end{itemize}
    \item \inlinebox{sudo cfdisk /dev/sdc}
      \begin{itemize}
        \item GPT wählen
        \item DOS ist eine andere Bezeichnung für MBR
      \end{itemize}
    \item \inlinebox{sudo mkfs.ext4 /dev/sdc1}
      \begin{itemize}
        \item \inlinebox{-n 'My_USB'} to give it a name
      \end{itemize}
    \item \inlinebox{sudo chmod 755 . -R}
    \item \inlinebox{sudo chown areo:users . -R}
  \end{itemize}
  \begin{Sidenote}
    \begin{itemize}
      \item need to \inlinebox{sudo umount /dev/sdX} flash drive before \inlinebox{wipefs} / \inlinebox{mkfs.vfat} etc.
    \end{itemize}
  \end{Sidenote}
\end{frame}
