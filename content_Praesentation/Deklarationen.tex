%!Tex Root = ../Vorlage_Praesentation.tex
% ./Packete.tex
% ./Design.tex
% ./Kapitel1.tex
% ./Kapitel2.tex

% ┌──────────┐
% │ Settings │
% └──────────┘

% https://tex.stackexchange.com/questions/584071/configure-minted-style-in-latex-for-code-highlighting
% https://pygments.org/docs/tokens/
% https://pygments.org/docs/styledevelopment/#creating-own-styles
% 'custom':  'custom::CustomStyle' in /usr/lib/python3.10/site-packages/pygments/styles/__init__.py
\usemintedstyle{custom}

% https://tex.stackexchange.com/questions/345976/global-settings-for-minted
\setminted{fontsize=\tiny,breaklines,highlightcolor=SecondaryColorDimmed,autogobble,escapeinside=||,breakafter={_},breakbefore={(},breakaftersymbolpre={},breakaftersymbolpost={},breakbeforesymbolpre={},breakbeforesymbolpost={},breaksymbolsepleft=2mm,breaksymbolsepright=0mm,breakindent=0mm,breaksymbolindentleft=5mm,breaksymbolindentright=0mm,numbersep=2.5mm}

\newenvironment{linenums}{
  \setminted{linenos}
}{}

\setlist[itemize]{itemsep=2mm, parsep=0mm, partopsep=0mm, topsep=0.25cm}

% ┌─────────────────┐
% │ Classical Boxes │
% └─────────────────┘

\DeclareTCBInputListing{\codebox}{ s o m }{listing file={#3},
  enhanced,colframe=PrimaryColor,colback=BoxColor,IfBooleanTF={#1}{colframe=SecondaryColor}{colframe=PrimaryColor},fonttitle=\tiny,#2,listing only,halign title=center,drop fuzzy shadow,arc=1mm,bottom=1mm,top=1mm,left=1mm,right=1mm,boxrule=0.5mm,listing engine=minted}
% , sharpish corners

% % https://tex.stackexchange.com/questions/585582/inside-of-a-newtcbinputlisting-how-can-i-change-the-color-of-the-line-number-as
% % https://www.overleaf.com/learn/latex/Using_colours_in_LaTeX
% https://tex.stackexchange.com/questions/132849/how-can-i-change-the-font-size-of-the-number-in-minted-environment
\renewcommand{\theFancyVerbLine}{\ttfamily\textcolor{white}{\tiny{\arabic{FancyVerbLine}}}}
% \renewcommand{\theFancyVerbLine}{\sffamily \textcolor[rgb]{0.5,0.5,1.0}{\huge \oldstylenums{\arabic{FancyVerbLine}}}}

\newtcbinputlisting{\numberedcodebox}[2][]{
  listing file={#2}, enhanced, colframe=PrimaryColor,colback=BoxColor, fonttitle=\small, #1, listing only, halign title=center,drop fuzzy shadow,arc=1mm,boxrule=0.5mm,listing engine=minted,overlay={\begin{tcbclipinterior}\fill[PrimaryColor] (frame.south west) rectangle ([xshift=4mm]frame.north west);\end{tcbclipinterior}}
}

\DeclareTColorBox{codeframe}{ s o m }{
  enhanced, halign title=center, fonttitle=\tiny, interior style={fill=white}, IfBooleanTF={#1}{frame style={color=SecondaryColor}}{frame style={color=PrimaryColor}}, title={#3}, #2,drop fuzzy shadow,arc=1mm,bottom=1mm,top=1mm,left=1mm,right=1mm,boxrule=0.5mm,listing engine=minted,minted style=colorful}

\newtcolorbox{file}[1][]{enhanced, hbox, notitle, interior style={fill=PrimaryColor}, frame empty, halign=center, fontupper=\color{white}\tiny, #1,drop fuzzy shadow,arc=1mm,bottom=1mm,top=1mm,left=1mm,right=1mm,boxrule=0.5mm,listing engine=minted,minted style=colorful}

\newtcblisting{terminal}{
enhanced,colframe=PrimaryColor,colback=BoxColor,hbox,listing only,halign title=center,minted language=text,drop fuzzy shadow,arc=1mm,bottom=1mm,top=1mm,left=1mm,right=1mm,boxrule=0.5mm,listing engine=minted,minted style=colorful,minted options={}}

% % https://tex.stackexchange.com/questions/593218/nested-inline-math-for-new-command-with-argument
\newcommand{\prompt}{\textcolor{SecondaryColor}{\setBold >\;\ignorespaces}}
% % https://tex.stackexchange.com/questions/593218/nested-inline-math-for-new-command-with-argument
\newcommand{\customprompt}{\textnormal\bfseries\textcolor{SecondaryColor}{S}\textcolor{gray!90!black}{he}\textcolor{SecondaryColorDimmed}{ll}\textcolor{SecondaryColor}{>}\;\ignorespaces}

\DeclareTotalTCBox{\inlinebox}{ s v }
{verbatim,colback=SecondaryColorDimmed,colframe=SecondaryColor}
{\IfBooleanTF{#1}%
{\textcolor{SecondaryColor}{\setBold >\enspace\ignorespaces}#2}%
{#2}}

\DeclareTotalTCBox{\key}{ m }
{verbatim,colback=SecondaryColorDimmed,colframe=SecondaryColor}
{$\mathtt{#1}$}

\numberwithin{equation}{section}

% ┌──────────────┐
% │ New Commands │
% └──────────────┘

% alternative alert
\newcommand\aalert[1]{\textcolor{PrimaryColor}{#1}}

\newcommand{\coloritem}{%
\item[\textcolor{SecondaryColor}{$\blacktriangleright$}]\ignorespaces}

\newcommand{\colorlabel[1]}{%
\item[\textcolor{SecondaryColor}{$\blacktriangleright$}]\Alert{#1:}\enspace\ignorespaces}

\newcommand{\greyitem}{%
\item[\textcolor{gray!90!black}{$\blacktriangleright$}]\ignorespaces}

% https://tex.stackexchange.com/questions/329990/how-do-i-change-the-color-of-itemize-bullet-specific-and-default
% \renewcommand{\labelitemi}{$\textcolor{PrimaryColor}{\bullet}$}
% \renewcommand{\labelitemii}{$\textcolor{PrimaryColor}{\cdot}$}
% \renewcommand{\labelitemiii}{$\textcolor{PrimaryColor}{\diamond}$}
% \renewcommand{\labelitemiv}{$\textcolor{PrimaryColor}{\ast}$}

\newcommand{\notebook}{\hfill\includegraphics[height=3mm]{./figures/note.png}}

\newcommand{\magnifier}{\hfill\includegraphics[height=3mm]{./figures/lupe.png}}

\newcommand\RemoveMargin[2][3em]{%
\makebox[\linewidth][c]{%
  \begin{minipage}{\dimexpr\textwidth+#1\relax}
  \raggedright#2
  \end{minipage}%
  }%
}
