%!Tex Root = ../Vorlage_Praesentation.tex
% ./content_Praesentation/Kapitel2.tex

\if\hide1\section{Kapitel 1}\fi

\begin{frame}[label={kapitel1]{Kapitel 1}{Beispiel für Tabelle}
  \scriptsize
  \begin{table}[H]
    \center
    \begin{NiceTabular}{X[1,c]X[2,l]}[rules/color=PrimaryColor] % {\linewidth}{|C|C|L|L|}
      \CodeBefore
      \chessboardcolors{white}{BoxColor}
      \rowcolor{PrimaryColor}{1}
      \Body
      \textcolor{white}{Spalte 1} & \textcolor{white}{Spalte 2} \\
      \scripttt{row1} & \alert{Beschreibung} für \scripttt{row1}. \\
      \scripttt{row2} & \Alert{Beschreibung} für \scripttt{row2}. \\
      \bottomrule
    \end{NiceTabular}
    \caption{Beispiel.}
  \end{table}
\end{frame}
