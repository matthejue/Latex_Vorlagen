%!Tex Root = ../Vorlage_Praesentation.tex
% ./Packete.tex
% ./Design.tex
% ./Deklarationen.tex
% ./Kapitel1.tex

\if\hide0\section{Kapitel 2}\fi

\begin{frame}[fragile]{Kapitel 1}{Beispiele für Codeboxen}
  \begin{columns}
    \begin{column}{0.4\textwidth}
      \codebox[title=Codebeispiel, remember as=codebeispiel1, minted language=c, minted options={highlightlines={1,3-7}}]{./code_examples/example.c}
    \end{column}
    \hfill
    \begin{column}{0.4\textwidth}
      \begin{linenums}
        \numberedcodebox[remember as=codebeispiel2, minted language=c, minted options={}]{./code_examples/example.c}
      \end{linenums}
    \end{column}
  \end{columns}
  \begin{tikzpicture}[overlay,remember picture,line width=1mm,draw=PrimaryColor, font=\small, draw=SecondaryColor]
    \draw[->] (codebeispiel1.east) to[bend left] node[above, align=center] {Mehrzeiliger\\ Text} (codebeispiel2.west);
  \end{tikzpicture}
\end{frame}

\begin{frame}[fragile, label={kapitel2}]{Kapitel 2}{Beispiele für Inline Boxen}
    \begin{itemize}
      \coloritem \inlinebox{arg} aus \inlinebox*{command arg}.
      \coloritem \key{\Uparrow-Tab}.
    \end{itemize}
\end{frame}

\begin{frame}{Kapitel 2}{Beispiel für hierarchisch aufgebaute Codebeispiele}
  \centering
  \begin{file}[remember as=input]
    Start
  \end{file}
  \vspace{0.1cm}
  \begin{codeframe}*[remember as=frameone]{Frame 1}
    \codebox[title=Codebeispiel 1, width=0.4\linewidth, nobeforeafter, minted language=text, equal height group=A, minted options={highlightlines={2}}]{./code_examples/example2.special}
    \hfill
    \codebox*[title=Codebeispiel 2, width=0.4\linewidth, nobeforeafter, minted language=text, equal height group=A, minted options={highlightlines={3}}]{./code_examples/example2.special}
  \end{codeframe}
  \vspace{0.1cm}
  \begin{file}[remember as=conta]
    ...
  \end{file}
  \begin{tikzpicture}[overlay,remember picture,line width=1mm,draw=PrimaryColor, font=\tiny]
    \draw[->, color=SecondaryColor] (input.south) to node[right=0.25cm, align=center] {\textcolor{black}{Verbindung 1}} (frameone.north);
    \draw[->] (frameone.south) to node[right=0.25cm, align=center] {Verbindung 2} (conta.north);
  \end{tikzpicture}
\end{frame}

\begin{frame}[fragile]{Kapitel 2}{Beispiel für Terminal}
  \centering
  \begin{terminal}
    |\prompt| command arg
    |\customprompt| command arg
  \end{terminal}
  %   % |\prompt|
\end{frame}
