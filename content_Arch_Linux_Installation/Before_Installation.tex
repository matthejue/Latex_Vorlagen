%!Tex Root = ../Arch_Linux_Installation.tex
% ./content_Arch_Linux_Installation/Packete
% ./content_Arch_Linux_Installation/Design
% ./content_Arch_Linux_Installation/Deklarationen
% ./content_Arch_Linux_Installation/Base_Installation
% ./content_Arch_Linux_Installation/After_Base_Installation

\section{Before Installation}

\begin{frame}[allowframebreaks]{Before Installation - 1}
  \begin{itemize}
    \item in der UEFI firmware fast-boot auf [Disabled] setzen.
    \item \enquote{Schnellstart} in Windows deaktivieren, da die EFI Systempartition beschädigt werden kannn.
      \begin{enumerate}
        \item Windows-Taste + X drücken / Systemsteuerung starten.
        \item Hier nun System und Sicherheit / Energieoptionen starten.
        \item Links nun \enquote{Auswählen, was beim Drücken des Netzschalters geschehen soll} anklicken.
        \item Im neuen Fenster nun oben auf: Einige Einstellungen sind momentan nicht verfügbar anklicken.
        \item Nun wird unten bei \enquote{Einstellungen für das Herunterfahren} der Haken bei Schnellstart aktivieren (Empfohlen) anklickbar. Nun kann man den Haken entweder entfernen oder setzen.
      \end{enumerate}
    \item use Belena Etcher (`sudo balena-etcher-electron`) to put the `.iso` on a usb-device, then go into UEFI firmware settings (start with e.g. f2 during system start, `systemctl reboot --firmware-setup` or select it in GRUB with the option `UEFI firmware settings`) and change the bootorder to have the usb-device having a higher boot priority then the esp partition with it's bootloader that is usually loaded.
    \item maybe disable secure boot
  \end{itemize}
\end{frame}
