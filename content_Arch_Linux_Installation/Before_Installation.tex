%!Tex Root = ../Arch_Linux_Installation.tex
% ./content_Arch_Linux_Installation/Packete
% ./content_Arch_Linux_Installation/Design
% ./content_Arch_Linux_Installation/Deklarationen
% ./content_Arch_Linux_Installation/Base_Installation
% ./content_Arch_Linux_Installation/After_Base_Installation

\section{Before Installation}

\begin{frame}[fragile, allowframebreaks]{Before Installation}
  \begin{itemize}
    \item in der UEFI firmware fast-boot auf [Disabled] setzen.
    \item \enquote{Schnellstart} in Windows deaktivieren, da die EFI Systempartition beschädigt werden kannn.
      \begin{enumerate}
        \item Windows-Taste + X drücken / Systemsteuerung starten.
        \item Hier nun System und Sicherheit / Energieoptionen starten.
        \item Links nun \enquote{Auswählen, was beim Drücken des Netzschalters geschehen soll} anklicken.
        \item Im neuen Fenster nun oben auf: Einige Einstellungen sind momentan nicht verfügbar anklicken.
        \item Nun wird unten bei \enquote{Einstellungen für das Herunterfahren} der Haken bei Schnellstart aktivieren (Empfohlen) anklickbar. Nun kann man den Haken entweder entfernen oder setzen.
      \end{enumerate}
    \item use Belena Etcher (\inlinebox{sudo balena-etcher-electron}) to put the \inlinebox{.iso} on a usb-device, then go into UEFI firmware settings (start with e.g. f2 during system start, \inlinebox{systemctl reboot --firmware-setup} or select it in GRUB with the option \inlinebox{UEFI firmware settings}) and change the bootorder to have the usb-device having a higher boot priority then the esp partition with it's bootloader that is usually loaded.
    \item maybe disable secure boot
  \end{itemize}
\end{frame}

\begin{frame}[fragile, allowframebreaks]{Create Installmedia}
  \begin{itemize}
    \item \alert{on Linux:} Balena-Etcher, easiest way to download AppImage und use AppimageManager.
      % \item Poolkit Gnome vielleicht erwähnen
    \item \alert{on Windows:} Rufus.
  \end{itemize}
\end{frame}

\begin{frame}[fragile]{Start UEFI with boot key}
  \begin{itemize}
    \item \aalert{Dell:} \key{F2} or \key{F12}.
    \item \aalert{HP:} \key{ESC} or \key{F10}.
    \item \aalert{Acer:} \key{F2} or \key{Delete}.
    \item \aalert{ASUS:} \key{F2} or \key{Delete}.
    \item \aalert{Lenovo:} \key{F1} or \key{F2}.
  \end{itemize}
\end{frame}

\begin{frame}[fragile, allowframebreaks]{Start UEFI from Windows}
  \begin{itemize}
    \item Settings $\Rightarrow$ Update \& Security $\Rightarrow$ Recovery $\Rightarrow$ Restart Now $\Rightarrow$ Troubleshoot Advanced Options $\Rightarrow$ UEFI Firmware Settings $\Rightarrow$ Restart.
  \end{itemize}
  % https://www.windowscentral.com/how-enter-uefi-bios-windows-10-pcs
\end{frame}
%
\begin{frame}[fragile]{Start UEFI from Linux}
  \begin{itemize}
    \item \inlinebox{systemctl reboot --firmware-setup}.
  \end{itemize}
  % https://superuser.com/questions/519718/linux-on-uefi-how-to-reboot-to-the-uefi-setup-screen-like-windows-8-can
\end{frame}

\begin{frame}[fragile]{Windows Partitionierung}
  \begin{itemize}
    \item asdf
  \end{itemize}
  % https://medium.com/linuxforeveryone/how-to-install-ubuntu-20-04-and-dual-boot-alongside-windows-10-323a85271a73
\end{frame}

\begin{frame}[fragile]{Größen von Partitionen durch 10 ganzzahlig teilbar}
  \begin{itemize}
    \item asdf
  \end{itemize}
\end{frame}

\section{Ubuntu Base Installation}

\begin{frame}[fragile]{Ubuntu Manuelle Partitionierung}
  \begin{itemize}
    \item
  \end{itemize}
  % https://medium.com/linuxforeveryone/how-to-install-ubuntu-20-04-and-dual-boot-alongside-windows-10-323a85271a73
\end{frame}

% später noch über Polkit Terminal und Graphical sprechen
% später noch wie Lan aktivieren
