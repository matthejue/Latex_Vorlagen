%!Tex Root = ../Main.tex
% ./Packete_und_Deklarationen.tex
% ./Motivation.tex
% ./Einführung.tex
% ./Implementierung1_Tables_DT_AST.tex,
% ./Implementierung2_Pntr_Array.tex,
% ./Implementierung3_Struct_Derived.tex,
% ./Implementierung4_Fun.tex,
% ./Implementierung5_Fehlermeldungen.tex,
% ./Ergebnisse_und_Ausblick.tex
% ./Appendix.tex

\newtcolorbox{titlebox}[1]{skin=enhanced, arc=0mm, boxrule=0mm,
    title style={preaction={fill=PrimaryColor}, pattern=fivepointed stars, pattern color=white, opacity=0.1},
    interior style={preaction={fill=SecondaryColor}, pattern=fivepointed stars, pattern  color=white, opacity=0.3},
    frame style={color=white},
    % segmentation style={black,solid,opacity=0.2,line width=1pt}
    title={#1}
}

\begin{titlepage}
  \vspace{1cm}
  \center
  \textsc{\LARGE Albert Ludwigs Universität Freiburg}\\[0.5cm]
  \textsc{\Large Technische Fakultät}\\[2.0cm]

  \vspace{1cm}

  \begin{titlebox}{\center \huge \bfseries PicoC-Compiler}
    \center
    % \\
    % \tcblower
    {\bfseries \center \LARGE \setstretch{1.1} Übersetzung einer Untermenge von C in den Befehlssatz der RETI-CPU\par}
  \end{titlebox}
    \textsc{\large Bachelorarbeit}\\
    \rule{\linewidth}{0.1mm}

  % {\large \emph{Abgabedatum:} 13. September 2022}\\[2.5cm]
  \vspace{3cm}

  \begin{minipage}{0.45\textwidth}
    \begin{flushleft} \large
      \emph{Autor:}\\
      Jürgen Mattheis\\
      \hspace{1cm}\\
      \hspace{1cm}\\
      \hspace{1cm}\\
      \hspace{1cm}
    \end{flushleft}
  \end{minipage}
  ~
  \begin{minipage}{0.45\textwidth}
    \begin{flushright} \large
      \emph{Gutachter:}\\
      Prof. Dr. Scholl\\[0.64cm]
      \emph{Betreuung:}\\
      M.Sc. Seufert\\
    \end{flushright}
  \end{minipage}

  \vspace{8.5cm}
  \rule{11cm}{0.1mm}\\[0.25cm]
  \large{Eine Bachelorarbeit am Lehrstuhl für}\\
  \large{Betriebssysteme}
\end{titlepage}
