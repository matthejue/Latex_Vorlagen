%!Tex Root = ../Vorlage_Uebungsblatt.tex
% ./Packete.tex
% ./Deklarationen.tex
% Kapitel1
% Kapitel2

% ┌─────────┐
% │ Setings │
% └─────────┘

% https://tex.stackexchange.com/questions/329990/how-do-i-change-the-color-of-itemize-bullet-specific-and-default
\renewcommand{\labelitemi}{$\textcolor{PrimaryColor}{\blacktriangleright}$}
\renewcommand{\labelitemii}{$\textcolor{PrimaryColor}{\blacktriangleright}$}
\renewcommand{\labelitemiii}{$\textcolor{PrimaryColor}{\blacktriangleright}$}
\renewcommand{\labelitemiv}{$\textcolor{PrimaryColor}{\blacktriangleright}$}
% \bullet, \cdot, \diamond, \ast

% ------------------------------ Minted Settings ------------------------------

% https://tex.stackexchange.com/questions/584071/configure-minted-style-in-latex-for-code-highlighting
% https://pygments.org/docs/tokens/
% https://pygments.org/docs/styledevelopment/#creating-own-styles
% 'custom':  'custom::CustomStyle' in /usr/lib/python3.10/site-packages/pygments/styles/__init__.py
% https://tex.stackexchange.com/questions/18083/how-to-add-custom-c-keywords-to-be-recognized-by-minted#comment930474_42392
\usemintedstyle{custom}

% https://tex.stackexchange.com/questions/345976/global-settings-for-minted
\setminted{fontsize=\normalsize,breaklines,highlightcolor=SecondaryColor,autogobble,escapeinside=||,breakafter={_},breakbefore={(},breakaftersymbolpre={},breakaftersymbolpost={},breakbeforesymbolpre={},breakbeforesymbolpost={},breaksymbolsepleft=2mm,breaksymbolsepright=0mm,breakindent=0mm,breaksymbolindentleft=5mm,breaksymbolindentright=0mm,numbersep=2mm}


\newenvironment{linenums}{
  \setminted{linenos}
}{}

% https://tex.stackexchange.com/questions/7210/label-and-caption-without-float
\DeclareCaptionType{codecaption}[Code][Codeverzeichnis]
% https://tex.stackexchange.com/questions/449677/spaces-in-newenvironment
\newenvironment{code}{\bigskip\captionsetup{type=codecaption}}{\medskip}

% https://tex.stackexchange.com/questions/585582/inside-of-a-newtcbinputlisting-how-can-i-change-the-color-of-the-line-number-as
% https://www.overleaf.com/learn/latex/Using_colours_in_LaTeX
\renewcommand{\theFancyVerbLine}{\ttfamily\textcolor{white}{\normalsize{\arabic{FancyVerbLine}}}}

% ┌────────────────────────────────────────┐
% │ Boxes and everything connected with it │
% └────────────────────────────────────────┘

% https://tex.stackexchange.com/questions/538528/tcolorbox-newtcbtheorem-index-with-tcolorbox
\newtcbtheorem[list inside={theorem_list},crefname={definition}{definitions}, number within=chapter]{Definition}{Definition}%
{enhanced, breakable, arc=0mm,top=3mm,bottom=3mm, boxrule=0mm, borderline south={1mm}{0pt}{PrimaryColor}, title style={color=PrimaryColor},
interior style={opacity=0.2, fill=PrimaryColor},
frame style={color=white}, fonttitle=\bfseries, fontupper=\itshape,
before upper=\setlength{\parskip}{1em},
after title={\hfill\includegraphics[height=3mm]{./figures/note.png}}
}{def}

\DeclareTCBInputListing{\codebox}{ s o m }{listing file={#3},
  enhanced,colframe=PrimaryColor,colback=BoxColor,IfBooleanTF={#1}{colframe=SecondaryColor}{colframe=PrimaryColor},fonttitle=\bfseries\normalsize,#2,listing only,halign title=center,drop fuzzy shadow,arc=1mm,bottom=1mm,top=1mm,left=1mm,right=1mm,boxrule=0.5mm,listing engine=minted}
% , sharpish corners

% % https://tex.stackexchange.com/questions/585582/inside-of-a-newtcbinputlisting-how-can-i-change-the-color-of-the-line-number-as
% % https://www.overleaf.com/learn/latex/Using_colours_in_LaTeX
% https://tex.stackexchange.com/questions/132849/how-can-i-change-the-font-size-of-the-number-in-minted-environment
\renewcommand{\theFancyVerbLine}{\ttfamily\textcolor{white}{\normalsize{\arabic{FancyVerbLine}}}}
% \renewcommand{\theFancyVerbLine}{\sffamily \textcolor[rgb]{0.5,0.5,1.0}{\huge \oldstylenums{\arabic{FancyVerbLine}}}}

\newtcbinputlisting{\numberedcodebox}[2][]{
  listing file={#2}, enhanced, colframe=PrimaryColor,colback=BoxColor, fonttitle=\bfseries\normalsize, #1, listing only, halign title=center,drop fuzzy shadow,arc=1mm,boxrule=0.5mm,listing engine=minted,overlay={\begin{tcbclipinterior}\fill[PrimaryColor] (frame.south west) rectangle ([xshift=4mm]frame.north west);\end{tcbclipinterior}}
}

\DeclareTColorBox{codeframe}{ s o m }{
  enhanced, halign title=center, fonttitle=\bfseries\normalsize, interior style={fill=white}, IfBooleanTF={#1}{frame style={color=PrimaryColor}}{frame style={color=PrimaryColor}}, title={#3}, #2,drop fuzzy shadow,arc=1mm,bottom=1mm,top=1mm,left=1mm,right=1mm,boxrule=0.5mm,listing engine=minted,minted style=colorful}

\newtcolorbox{file}[1][]{enhanced, hbox, notitle, interior style={fill=SecondaryColor}, frame empty, halign=center, fontupper=\color{white}\normalsize, #1,drop fuzzy shadow,arc=1mm,bottom=1mm,top=1mm,left=1mm,right=1mm,boxrule=0.5mm,listing engine=minted,minted style=colorful}

\newtcblisting{terminal}{
enhanced,colframe=PrimaryColor,colback=BoxColor,hbox,listing only,halign title=center,minted language=text,drop fuzzy shadow,arc=1mm,bottom=1mm,top=1mm,left=1mm,right=1mm,boxrule=0.5mm,listing engine=minted,minted style=colorful,minted options={}}

% % https://tex.stackexchange.com/questions/593218/nested-inline-math-for-new-command-with-argument
\newcommand{\prompt}{\textcolor{PrimaryColor}{\setBold >\;\ignorespaces}}
% % https://tex.stackexchange.com/questions/593218/nested-inline-math-for-new-command-with-argument
\newcommand{\customprompt}{\textnormal\bfseries\textcolor{PrimaryColor}{S}\textcolor{gray!90!black}{he}\textcolor{SecondaryColor}{ll}\textcolor{PrimaryColor}{>}\;\ignorespaces}

\DeclareTotalTCBox{\inlinebox}{ s v }
{verbatim,colback=SecondaryColorDimmed,colframe=SecondaryColor}
{\IfBooleanTF{#1}%
{\textcolor{SecondaryColor}{\setBold >\enspace\ignorespaces}#2}%
{#2}}

\DeclareTotalTCBox{\key}{ m }
{verbatim,colback=SecondaryColorDimmed,colframe=SecondaryColor}
{$\mathtt{#1}$}

\newtcolorbox{Sidenote}{enhanced,breakable,drop fuzzy shadow,sharpish corners, notitle,arc=0mm,left=3mm,right=3mm,boxrule=1mm, borderline vertical={1mm}{0pt}{PrimaryColor},title=Sidenote,attach boxed title to top text left={yshift=0mm},
interior style={fill=BoxColor},
frame style={color=BoxColor},
% https://tex.stackexchange.com/questions/459870/paragraph-breaks-inside-tcolorbox, maybe also parbox=false
boxed title style={arc=0mm,skin=enhancedfirst jigsaw,frame empty,colback=PrimaryColor,boxrule=0mm,bottom=-0.4mm},
after title={\hspace{0.2cm}\includegraphics[height=3mm]{./figures/lupe.png}}
}
% before upper=\setlength{\parskip}{1em},
% fonttitle=\bfseries

\numberwithin{equation}{section}

% ┌───────────────────┐
% │ Header and Footer │
% └───────────────────┘

\fancypagestyle{default}{%
  \fancyhf{}
  \fancyfootoffset{1cm}
  \fancyheadoffset{1cm}
  \fancyhead[R]{\nouppercase{\rightmark}}
  \fancyhead[L]{\nouppercase{\leftmark}}
    \renewcommand{\headrulewidth}{0.2mm}
  \fancyfoot[R]{\thepage}
}

\fancypagestyle{plain}{%
    \fancyhf{}
    \renewcommand{\headrulewidth}{0mm}
}

\fancypagestyle{bib}{%
    \fancyhf{}
    \fancyfootoffset{1cm}
    \renewcommand{\headrulewidth}{0mm}
    \fancyfoot[R]{\thepage}
}

% ┌──────────────────────────────────────────────┐
% │ Space after and before Section, Chapter etc. │
% └──────────────────────────────────────────────┘

% \titlespacing*{\section}
\titlespacing*{\section}{0cm}{*4}{*4}
\titlespacing*{\subsection}{0cm}{*3}{*3}
\titlespacing*{\subsubsection}{0cm}{*2}{*2}

% https://tex.stackexchange.com/questions/139366/chapter-header-with-super-huge-numbers
\usepackage{fix-cm}
\newcommand{\hsp}{\hspace{0pt}}
\titleformat{\chapter}[hang]{\bfseries}{\fontsize{100}{0}\selectfont \textcolor{PrimaryColor}\thechapter\hsp}{0.5cm}{\Huge}[]

% https://tex.stackexchange.com/a/40001
\titlespacing*{\chapter}{0cm}{*0}{*4}

% https://stackoverflow.com/questions/2854299/getting-subsection-to-list-in-table-of-contents-in-latex
\setcounter{tocdepth}{4}

% https://tex.stackexchange.com/questions/186981/is-there-a-subsubsubsection-command
% https://tex.stackexchange.com/questions/130795/how-can-i-number-sections-below-subsection-in-latex
\setcounter{secnumdepth}{5}
% https://tex.stackexchange.com/questions/32160/new-line-after-paragraph
\newcommand{\newlineparagraph}[1]{\paragraph{#1}\mbox{}\\\vspace{-0.5cm}}

\newcommand{\newplain}{
  \fancypagestyle{plain}{%
      \fancyhf{}%
      \fancyfootoffset{1cm}
      \renewcommand{\headrulewidth}{0mm}%
      \fancyfoot[R]{\thepage}
  }
}
